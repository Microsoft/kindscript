\section{USB Flashing Format}
\label{sec:uf2}

The way that code gets from a host computer onto a microcontroller is deeply rooted in 1980's technologies - 
serial wires, obscure protocols, and text-based file formats with limited line length. Depending on exact circumstances,
one must install serial USB drivers, select the right port and parameters, and then use a native application to access
the microcontroller, as the browser is not allowed to. As the advance of maker content in educational curricula continues,
this complicated flashing process presents one of the major obstacles to adoption in schools, where installing any software
is usually the domain of IT administrators.

One solution would be to rely on emerging standards, like WebUSB and WebBluetooth to transfer a program from the browser 
to the microcontroller. These standards are however still in their infancy, and it may take even longer before they are 
deployed in schools.

Another solution was pioneered by ARM with its DAPLink firmware, where a USB-capable microcontroller presents itself 
to an external computer as a USB pen drive. No special drivers need to be installed, as operating systems support pen
drives out of the box, typically formatted using FAT. The USB pen drive protocol (Mass Storage Class or MSC) is a
block-level protocol (generally 512 bytes) with no concept of files. DAPLink exposes a virtual FAT file system, which
is very small and never changes due to OS writes - it has an informational text file and an HTML file with a redirect
to the online editing environment. Otherwise, the FAT and the root directory table are empty. When the OS tries to
read a block, the DAPLink computes what should be there dynamically, based on compiled-in contents of the info and HTML files.

On file system writes, DAPLink detects the HEX format, and flashes it into the target microcontroller's memory,
over debug wires. Let's consider the problem that DAPLink must solve. It sees a 512 byte block of data to be written
at a particular block index on the device. It needs to decide if it's part of the file being flashed and if so, extract
the data and write it to the target microcontroller. Furthermore, when the OS writes a HEX file, DAPLink needs to discard
writes to the FAT or directory table, as well as writes of the meta-data files. It may need to deal with out-of-order writes.
All these details mean that DAPLink is quite complex and sometimes needs to be updated when a new OS release changes the way
in which it handles FAT. This also is the reason that some MSC bootloaders for various chips only support given operating
systems under some specific conditions.

The task would be simplified then, if every 512 block of the file being flashed was easy to distinguish from meta-data
or other random files, and easy to act on, independent of other blocks. The HEX file format doesn't give us these properties 
- the 512 byte boundary can be in the middle of a line in the HEX file, and even if we have an entire line, every line only
contains the last 16 bits of the address where to flash, with the upper 16 supplied only when they change.

For these reasons, we designed the USB Flashing Format (UF2). It consists of 512 byte blocks, where each block contains:
\begin{itemize}
\item magic numbers at the beginning and end (to heuristically distinguish it from any other data the OS writes);
\item the address in the target chip flash where the payload should be written;
\item the payload data (up to 476 bytes)
\end{itemize}

