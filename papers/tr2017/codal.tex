\section{The CODAL Runtime}
\label{sec:codal}

\emph{TBALL comment: the main focus on CODAL here is the support for event-based programming with concurrrent handlers,
   providing the bridge between scripting languages such as JavaScript and the world of microcontrollers. }

For reasons of efficiency and ease of access to hardware, code for microcontrollers is typically written in C/C++. 
However, this still leaves the programming model undefined. The Arduino programming model is based on polling: 
an Arduino ``sketch'' is a program template that consists of a setup procedure, for initialization of data structures,
and a loop procedure; programmers implement the body of setup and loop, where they explicitly poll the state of the
microcontroller’s pins. The Arduino model emphasizes sequential composition, which is not well-suited to reactive
systems where event arrival is unpredictable, and events must be handled quickly to achieve responsiveness.

The component-oriented device abstraction layer (CODAL) presents a programming model consisting of a set of
components that abstract away microcontroller details (such as pins and whether one polls or uses interrupts)
and communicate via events using a publish/subscribe mechanism.  Components can schedule event handlers to run
concurrently using a non-preemptive fiber scheduler. Each software component abstracts a hardware feature, sensor,
or connected device. \emph{TODO: how CODAL handles responsiveness?}

The CODAL runtime provides higher-level abstractions to the compiler developer than the Arduino runtime, making it
easy to compile event-based mechanisms, such as found in scripting languages like JavaScript, to the world of the
microcontroller.  [Greater flexibility, learning opportunities. We will show it is competitive in terms of memory
usage and performance.]

\subsection{Components}

\emph{TODO - CodalComponent: software model of device driver (physical and logical); components requiring periodic invocation}

\subsection{Events and the Message Bus}

A message bus enables a simple but powerful eventing model. Every runtime component has its own namespace for events
which can be raised at any time. Users may register event handlers for any number of such events. Moreover, event handlers
are transparently run in their own fiber - providing decoupling of user code from system code, device drivers and interrupts.
Fibers may also block awaiting events---a simple yet effective condition synchronisation mechanism.


\subsection{Fiber Scheduler}

A fiber scheduler enables a degree of managed concurrency on the device. Many high level languages require asynchronous
execution for responsiveness. Support for these languages can be greatly simplified through the presence of a scheduler. 
The scheduler is, by design, non-preemptive to minimize race conditions and interleaving of user code. This is also informed
by recent studies of concurrency in visual programming languages \cite{meerbaum2013learning}. A further benefit of running a 
scheduler is that when no user code needs to be executed, the device can safely be put into a power efficient sleep mode,
transparently to the user.

\subsection{CodalDevice}

\emph{Putting it all together}